\documentclass{article} 
\usepackage{polyglossia} 
\usepackage{amsmath}
\usepackage{fontspec} 
\usepackage{lipsum} 
\usepackage[margin=1in]{geometry}
\usepackage{graphicx} 
\usepackage{caption} 
\usepackage{subcaption}
\usepackage{hyperref} 
\usepackage{listing}
\hypersetup{% 
    colorlinks=true, linkcolor=blue, filecolor=magenta,      
    urlcolor=cyan, 
    pdfinfo = {%
        Title = Συστήματα Πολυμέσων Υλοποίηση GSM 06.10,
        Author = {Χρήστος Μάριος Περδίκης, Γιώργος Βακασίρης},
        Producer = XeLaTeX,
    } 
}

\setmainfont{FreeSans}


\title{Πλήρωση Πολυγώνων}
\date{Εαρινό Εξάμηνο 2024-2025}
\author{Χρήστος-Μάριος Περδίκης 10075 cperdikis@ece.auth.gr}

\begin{document}
\maketitle

Αυτή είναι η αναφορά για την πρώτη εργασία του μαθήματος Γραφική με 
Υπολογιστές, εαρινό εξάμηνο 2024-2025. Στόχος της η υλοποίηση αλγορίθμου
πλήρωσης τριγώνων και η χρωμάτισή τους με flat και texture map
shading. Ακολουθεί περιγραφή της υλοποίησης.

\section{Γενική δομή}
Η υλοποίηση της γραμμικής παρεμβολής βρίσκεται στο αρχείο ``vec\_inter.py''. 
Η υλοποίηση των αλγορίθμων πλήρωσης τριγώνων και χρωματισμού βρίσκεται στο
``polygon\_fill.py''. Στα αρχεία ``demo\_f.py'', ``demo\_g.py'' φορτώνονται 
τα δεδομένα των vertices, faces, κλπ. και καλείται η συνάρτηση \verb|render_img|.
Σε αυτή δημιουργείται μια νέα εικόνα λευκού background και διαστάσεων $512\times512$, υπολογίζεται το κέντρο 
βάρους βάθους των κορυφών όλων των τριγώνων, και ζωγραφίζονται 
ένα-ένα από τη συνάρτηση \verb|polygon_fill| κατά φθίνουσα σειρά βάθους τριγώνου.
Η \verb|polygon_fill| υλοποιεί τον αλγόριθμο πλήρωσης τριγώνων και καλεί τις 
συναρτήσεις χρωματισμού \verb|t_shading| και \verb|f_shading|. Όταν ολοκληρωθεί 
η τελική εικόνα, αυτή αποθηκεύεται σε αρχείο ``texture\_shading.png'' ή ``flat\_shading.png''
για texture map shading και flat shading αντίστοιχα.

\section{Γραμμική Παρεμβολή}
TODO

\section{Αλγόριθμος Πλήρωσης Πολυγώνων}
Ο αλγόριθμος πλήρωσης τριγώνων υλοποιείται στη συνάρτηση \verb|polygon_fill|,
με κάθε κλήσης της συνάρτησης πληρείται ένα τρίγωνο. Για κάθε πλευρά ενός 
τριγώνου υπολογίζονται τα $y_{max}$, $y_{min}$, $x_{max}$, $x_{min}$ 
και η κλίση $m_k$. Μετά αρχικοποιούνται οι λίστες ενεργών ακμών και ενεργών 
οριακών σημείων σημείων. Στην αρχικοποίηση αυτή το scanline έχει τιμή $y = min(y_{min}) - 1$.
Οι δύο λίστες είναι numpy arrays. Οποιοδήποτε στοιχείο του array ενεργών σημείων που δεν 
είναι $(-1, -1)$ είναι ενεργό. Αντίστοιχα οποιαδήποτε ακμή που δεν αποτελείται από 
δύο σημεία $(-1, -1)$ είναι ενεργή.

Η λούπα πλήρωσης τριγώνων ξεκινάει για $y = min(y_{min})$ και τερματίζει για
$y = max(y_{max})$. Για κάθε iteration της λούπας, η λίστα ενεργών οριακών 
σημείων ταξινομείται με αύξουσα σειρά κατά $x$. Λόγω της φύσης του αλγόριθμου 
εύρεσης ενεργών οριακών σημείων και ενεργών ακμών, μόνο δύο σημεία / ακμές 
μπορεί να είναι ενεργές ανά πάσα στιγμή (αυτό οφείλεται στην κυρτότητα του
τριγώνου). Άρα το πρώτο σημείο της ταξινομημένης λίστας ενεργών οριακών σημείων
θα είναι πάντα το υπολειπόμενο ανένεργο σημείο. Ανάλογα με την τιμή της μεταβλητής 
$shading$, καλείται η αντίστοιχη συνάρτηση χρωματισμού \verb|t_shading| ή 
\verb|f_shading| για τεταγμένη ίση με το τωρινό scanline και τετμημένες στο range 
μεταξύ των δύο ενεργών οριακών σημείων. Μετά ενημερώνονται οι λίστες 
ενεργών οριακών σημείων και ενεργών ακμών και ξεκινά το επόμενο iteration.

\subsection{Εύρεση Ενεργών Ακμών}
Για κάθε μια από τις 3 πλευρές ενός τριγώνου γίνονται οι εξής ενέργειες.
Αν η επόμενη θέση του scanline είναι ίση με την ελάχιστη τεταγμένη της πλευράς
$y_{min} = y + 1$, και η πλευρά δεν είναι οριζόντια $y_{max} = y_{min}$, τότε 
προσθέτουμε αυτήν την πλευρά στη λίστα ενεργών ακμών. Επειδή δεν είναι
γνωστό ποιά τετμημένη αντιστοιχεί στο $y_{max}$ και ποιά στο $y_{min}$
κάθε πλευράς, χρειάζεται η κλήση της συνάρτησης γραμμικής παρεμβολής.
Γίνεται παρεμβολή μεταξύ των δύο σημείων της πλευράς στις τεταγμένες
$y_{max}$ και $y_{min}$ και η συνάρτηση \verb|vector_inter| επιστρέφει
τις τετμημένες τους. Επειδή
όμως $y_{max}$ και $y_{min}$ είναι οι τεταγμένες των σημείων που
ορίζουν την πλευρά,
επιστρέφεται η αντίστοιχη τετμημένη αυτών των σημείων.
Τέλος, αν η επόμενη θέση του scanline είναι ίση με τη μέγιστη τεταγμένη
της πλευράς του τριγώνου $y_{max} = y + 1$, η πλευρά αφαιρείται 
από τη λίστα ενεργών ακμών και την αντικαθιστούν σημεία με τιμές $(-1, -1)$.

\subsection{Εύρεση Ενεργών Οριακών Σημείων}
Κάθε ενεργή ακμή θα έχει και ένα ενεργό οριακό σημείο.
Για κάθε μια από τις πλευρές του τριγώνου γίνονται οι εξής ενέργειες.
Αν για την επόμενη θέση του scanline ισχύει $y_{min} = y + 1$, 
τότε στη λίστα ενεργών οριακών σημείων προστίθεται το $y_{min}$
και η αντίστοιχη τετμημένη του (η εύρεση της τετμημένης όμοια με 
πριν, μέσω της συνάρτησης \verb|vector_inter|). Για οποιοδήποτε 
ήδη ενεργό σημείο το οποίο δεν ανήκει σε κατακόρυφη γραμμή, η τετμημένη
τους αυξάνεται κατά $1 / m_k$. Δεν είναι απαραίτητη η ενημέρωση 
της τεταγμένης τους, εφόσον αυτή πάντα θα ισούται με την τιμή του τωρινού scanline.
Τέλος, αν η επόμενη θέση του scanline υπερβαίνει τη μέγιστη τεταγμένη
της πλευράς του τριγώνου $y_{max} = y + 1$, ή η πλευρά είναι οριζόντια 
$y_{max} = y_{min}$, τότε το σημείο αφαιρείται από τη λίστα ενεργών
οριακών σημείων, τη θέση του αντικαθιστά σημείο με συντεταγμένες 
$(-1, -1)$.

\section{Flat Shading}

\section{Texture Map Shading}
TODO

\end{document}
